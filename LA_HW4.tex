\documentclass[12pt]{amsart}

\usepackage[margin=1.25in]{geometry}

\usepackage{comment}

\begin{document}

\begin{center}
{\bf 108A HW4 }\\
\end{center}
\rightline{Yuchen Wang,Luis Ramirez,Shengbo Zhang}
\textbf{Section 2} \\
\textbf{Exercise 2}\\
(a) Let $v_1=(1,2,1,0)$ and $v_2=(1,2,3,1)$ Then $v_2$ $\cdot$ $v_1$ =8 so that\\
\begin{center}
$v_2' = v_2 - \dfrac{v_2 \cdot v_1}{v_1 \cdot v_1}v_1 = 1/3(-1,-2,5,3)$
\end{center}
Normalizing the orthogonal vectors, we have 
$\left\{{1/\sqrt{6}(1,2,1,0), 1/\sqrt{39}(-1,-2,5,3)}\right\}$\\
\\
(b) Let $v_1=(1,1,0,0)$, $v_2=(1,-1,1,1)$ and $v_3=(-1,0,2,1)$ Then $v_1$ $\cdot$ $v_2$ =0, so $v_2' = v_2$ so that for $v_3$ we have\\
\begin{center}
$v_3' = v_3 - \dfrac{v_3 \cdot v_1}{v_1 \cdot v_1}v_1 - \dfrac{v_3 \cdot v_2}{v_2 \cdot v_2}v_2' = 1/2(-2,2,3,1)$
\end{center}
Normalizing the orthogonal vectors, we have 
$\left\{{1/\sqrt{2}(1,1,0,0), 1/2(1,-1,1,1), /\sqrt{18}(-2,2,3,1)}\right\}$\\
 \\
\textbf{Exercise 4}\\
By number 3 we know that $\langle f,g \rangle$ = 1/4 and $ \langle f,f \rangle$ = we notice that\\
\begin{center}
$g' = g - \dfrac{\langle f,g \rangle}{\langle f,f \rangle}f = t^2 - \dfrac{3}{4}t$
\end{center}
Since $||g'|| = 1/\sqrt{80}$, we see that\\
$\left\{{\sqrt{3}t, \sqrt{80}(t^2-\dfrac{3}{4}t)}\right\}$\\
\\
\textbf{Exercise 5}\\
Let $f=1, g =t, h=t^2$. Then we get,\\
\begin{center}
$g' = g - \dfrac{\langle f,g \rangle}{\langle f,f \rangle}f = t - 1/2$
\end{center}
and\\
\begin{center}
$h' = h - \dfrac{\langle h,f \rangle}{\langle f,f \rangle}f - \dfrac{\langle h,g' \rangle}{\langle g',g' \rangle}f = t^2 - t + 1/6$
\end{center}
Therefore, normalizing we have,\\
$\left\{{1,\sqrt{12}(t-1/2), \sqrt{180}(t^2-t+\dfrac{1}{6})}\right\}$\\
\\
\\
\textbf{Section 3} \\
\textbf{Exercise 1}\\
(e)2\\
(f)3\\
(g)3\\
\\
\textbf{Exercise 2}\\
Let f and g be linear maps associated to A and B respectively,\\
Then AB = f$\cdot$g.\\
and $K^a$ $\underrightarrow{\text{g}}$ $K^b$ $\underrightarrow{\text{f}}$ $K^c$\\
Im f$\cdot$g = f(Im g)\\
if y$\in$Im(f$\cdot$g), there must exists a x$\in$ $K^a$ such that f(g(x)) = y.\\
Hence y$\in$Im(f), and thus Im(f$\cdot$g) $\subset$ Im(f).\\
Hence rank(AB)$\leq$rank(A).\\
Now consider\\
$K^a$ $\underrightarrow{\text{g}}$ $Im(g)$ $\underrightarrow{\overline{f}}$ $K^c$\\
Here $\overline{f}$ is the restriction of f to Im(g).\\
Since dim Im($\overline{f}$) $\leq$ dim Im(g) and Im(f$\cdot$g) = Im($\overline{f}$).\\
We get the conclusion that rank(AB)$\leq$rank(B).\\
\\
\textbf{Exercise 3}
We define that each line of matrix $A$ is $A^n$ so that we let $x_1A^1+x_2A^2+...+x_nA^n = 0$, then we get a system of the form:
$$ f(x)=\left\{ 
\begin{aligned}
x_1a_{11}+x_2a_{12}+...x_na_{1n} &= & 0 \\
x_2a_{22}+ ... x{n-1}a{2n-1}+x_na_{2n}&=& 0 \\
..............\\
x_na_{nn} =  0
\end{aligned}
\right.
$$
Since $a_{nn} \neq 0$, we know that $x_n=0$. By induction that we can get that $x_{n-1} = 0$. And at last we get that $x_1=x_2=...=x_n=0$. Thus rank$A=n$.\\
\\
\textbf{Exercise 4}\\
(c) Since $(4,7,\pi)$ and $(2,-1,1)$ are linearly independent so that the $dimS = 1$. And we let $z = 1$, we get that $(\frac{\pi+11}{18},\frac{\pi+2}{9},1)$ is a solution of the original system. So this vector forms a basis for S.\\
(d) The matrix of the original system is:

$$\begin{pmatrix} 
1 & 1 & 1\\
1 & -1 &0\\
0 & 1 & 1 
\end{pmatrix}\\
$$
So its rank is 3, so the space of solutions of the system is reduced to the single element $\{0\}$



\end{document}