\documentclass[12pt]{amsart}

\usepackage[margin=1.25in]{geometry}

\usepackage{comment}

\begin{document}

\begin{center}
{\bf 108A HW3 }\\

\end{center}
\rightline{Yuchen Wang,Luis Ramirez,Shengbo Zhang}
\textbf{Section 3} \\
\textbf{Exercise 3}\\
\\
Let Y be the subspace of $\mathbb{R}^4$ consisting of all $x \in \mathbb{R}^4$ such that\\ 
$x_1 + 2x_2 =0$ and $x_3 - 15x_4 =0$\\
It can be seen that (-2,1,0,0) and (0,0,15,1) form a basis of Y. Thus $dim Y = 2$\\
\\
\textbf{Exercise 15}\\
\\
(a) P is linear because\\
$P(A+B)$ = ($(A+B)$ + $^t\!(A+B)$)/2\\
= ($A+^t\!A + B + ^t\!B$)/2 = $P(A)+P(B)$\\
If c is a scalar then'\\
$P(cA) = (cA + ^t\!(cA))/2$\\
= $c(A+^t\!A)/2$ = $cP(A)$\\
(b) By definition of skew-symmetric matrix, an $nxn$ matrix A is skew symmetric if $^t\!A=-A$. the kernel of P shows $P(A)=0$, hence $A+^t\!A=0$ which equals $^t\!A=-A$.\\
(c)If $Sym_n(K)$ is the set of $nxn$ matrices and $Sk_(K)$ is the set of skew symmetric matrices, then\\
$Mat_{nxn}(K)=Sym_n(K)+Sk_n(K)$,\\
Hence,\\
dim Ker P= $n^2- n(n+1)/2= n(n-1)/2$
\\
\\
\textbf{Section 4} \\
\textbf{Exercise 2}\\
\\
\noindent Because $DimV = Dim KerL + Dim ImL$ and $ImL$ is a subspace of W.\\
So $Dim ImL \leq Dim W$\\
Because $DimV > DimW$\\
So $DimV - DimImL > 0$ and then $DimKerL > 0$\\
\\
\textbf{Exercise 7}\\
\\
\noindent If we want to prove that $L$ is invertible, we need to prove that the kernel of $L$ is $\{O\}$ and $L$ is surjective.\\
The first step is to prove that the kernel of $L$ is $\{O\}$:\\
The kernel of $L$ is:
$$\left\{
\begin{aligned}
2x&+& y= 0 \\
3x&-&5y= 0 \\
\end{aligned}
\right.
$$ 
Which is equal to:
$$\left\{
\begin{aligned}
x= 0 \\
y= 0 \\
\end{aligned}
\right.
$$ 
which means the kernel of $L$ is $\{O\}$.\\
Then we will do our second step:\\
we know that $L(x,y) = (2x+y,3x-5y)$ so we can easily conclude that $DimImL = 2$. Because $DimImL$ is a subspace of $R^2$ and $DimIm = DimR^2 = 2$. So $ImL$ is equal to $R^2$ which means $L$ is surjective.\\
So according to two conclusions we got above we can prove that $L$ is invertible.\\
\\
\textbf{Exercise 9}\\
a. $I - L^2 = I$\\
hence $(I+L)(I+L)=I$,\\
The inverse of I-L is I+L\\
b. $I=L(-L-2)$\\
The inverse of L is -L-2\\
c. $I=(I-L)(I+L+L^2)$
The inverse of I-L is $I+L+L^2$\\
\\
\textbf{Exercise 17}\\
$I=A-A^2$\\
hence$I=A(I-A)$,\\
the inverse of A is I-A\\.
Generalize\\
$I-A^(n+1)=(I-A)(I+A+...+A^n)=I$



\end{document}
